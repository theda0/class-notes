\documentclass{article}

\usepackage[english]{babel}
\usepackage{amsmath}
\usepackage{amssymb}
\usepackage{amsthm}
\usepackage[letterpaper,top=2cm,bottom=2cm,left=3cm,right=3cm,marginparwidth=1.75cm]{geometry}
\usepackage{graphicx}
\usepackage[colorlinks=true, allcolors=blue]{hyperref}
\usepackage{fancyhdr}
\usepackage{tikz}
\usetikzlibrary{matrix}
\usepackage[most]{tcolorbox}
\usepackage{hyperref}
\usepackage{array}

\theoremstyle{definition}
\newtheorem{example}{Example}[section]

\theoremstyle{definition}
\newtheorem{definition}{Definition}[section]

\theoremstyle{remark}
\newtheorem*{remark}{Remark}

\newtcbtheorem{theo}%
	{Theorem}{colback = blue!5!white,colframe = blue!50!black!50!}{theorem}
	
\newtcbtheorem{lemm}%
	{Lemma}{colback = green!5!white,colframe = green!50!black!50!}{lemma}

\newtcbtheorem{clai}%
{Claim}{colback = blue!5!white,colframe = blue!50!black!50!}{claim}

\newtcbtheorem{coroll}%
	{Corollary}{colback = purple!5!white,colframe = purple!50!black!50!}{corollary}    

\newtheorem{theorem}{Theorem}[section]
\newtheorem{corollary}{Corollary}[theorem]
\newtheorem{lemma}[theorem]{Lemma}
\newtheorem{claim}[theorem]{Claim}

\pagestyle{fancy}
\newcommand\size{1}% distance of nodes from center
\newcommand{\R}{\mathbb{R}}
\newcommand{\C}{\mathbb{C}}
\newcommand{\Z}{\mathbb{Z}}
\newcommand{\N}{\mathbb{N}}
\newcommand{\Q}{\mathbb{Q}}
\newcommand{\mb}[1]{\mathbb{#1}}
\newcommand{\mc}[1]{\mathcal{#1}}
\newcommand{\mk}[1]{\mathfrak{#1}}
\newcommand{\un}{\cup}
\newcommand{\ic}{\cap}
\begin{document}
\textbf{Last Time: } Gave specific (co)-limits and abelian categories.\\

\textbf{Today: } Existence of (co)-limits, adjunctions.\\

\textbf{Announcements: } Haine will be away this Friday (again), so we will have office hours on Thursday, 3:30 to 4:30. (Yay, not after class this tiem, but I'll have to skip 128A again..), not office hours on 3/15.\\

\section{Existence of Limits}

\begin{definition}
	A category $C$ is finite if the set of $\text{Mor}(C)$ is a finite set. $C$ is essentially finite if $C \simeq$ finite category. A finite (co)limit is a (co)limit indexed by a finite category, e.g. equalizers, pullbacks, and finite products.
\end{definition}

Note this also says there is finitely many objects  (in bijection with the identity morphisms).

\begin{theo}{}{}
	Let $E$ be a category that has all equalizers and all products. 
	Then for any category $I$ and diagram $F: I \to E$, the limit $\lim_I F$ exists and is given by 
	\[\lim_I F \simeq eq \Big( \prod_{i \in \text{Obj(I)}}F(i) \rightrightarrows \prod_{f:x \to y \in \text{Mor}(I)}F(y)\Big)\]
	Here the components of thte two arrows at $f:x \to y \in \text{Mor}(I)$ are defined by 
	\[\alpha_f: \prod_{i \in \text{Obj}(i)}F(i) \xrightarrow{pr_y}F(y)\]
	\[\beta_f: \prod_{i \in \text{Obj}(i)}F(i) \xrightarrow{pr_x}F(x) \xrightarrow{F(f)} F(y)\]

\end{theo}

Note the above theorem holds when we say finite products and finite categories.

\begin{lemm}{}{}
	$I$ a cateogry, $E$ a category with $I$-shaped limits. 
	Then a choice of limit of each $I$-shaped diagram $I \to E$ assmebles into a functor
	\[\lim_I \text{Fun(I, E)} \to E\]
mapping $F \to \lim_I F$ which is unique up to unique ISO.
\end{lemm}
Slogan: Taking the limit is a functor!

\section{ Adjunctions}

\begin{definition}
	An adjunction consists of functor $L:C \rightleftarrows D: R$ together with a natural isomorphism (in both of the variables)
	\[\text{Hom}_D (L(c), d) \cong \text{Hom}_C(c, R(d))\]
\end{definition}

In this setting we say that $L$ is the left adjoint and $R$ is the right adjoint.
\begin{example}
	In Set we have 'currying'
	\[\text{Hom}_{\text{Set}}(X \times Y, Z) \cong \text{Hom}_{\text{Set}}(X, \text{Hom}_{\text{Set}}(Y,Z))\]
\end{example}
 
\begin{example}
	$A$ a ring, then 
\[\text{Hom}_{\text{Mod(A)}}(L \otimes_A M, N) \cong \text{Hom}_{\text{Mod(A)}}(L, \text{Hom}_{\text{A}}(M,N))\]

\end{example}

 \begin{example}
	 $A$ a ring and $S$ a set, denote $A^{(S)} = \bigoplus_S A$ the free $A-$module w basis of $S$.
	 and we have 
	 \[\text{Hom}_{\text{Mod(A)}}(A^{(S)}, M) \cong \text{Hom}_{\text{Set}}(S, M))\]
one of many examples of a free and forgetful adjunction

\end{example}
\begin{example}
	$A$ a ring, $S$ a set,, denote $A[S] = A[t_s|s \in S]$  the polynomial ring w/ variables in $S$.
	we have
	\[\text{Hom}_{\text{alg(A)}}(A[S], B) \cong \text{Hom}_{\text{Set}}(S, B))\]

\end{example}

\begin{example}
	((co)extension/restriction of scalars) Given a  ring hom $\phi: A \to B$, we can realize $B$ as an $A$-module.\\
	\indent	(1) Extension of Scalars: $\phi^*: B \otimes_A (-): \text{Mod}(A) \to \text{Mod}(B)$.\\
	\indent (2) Restriction of Scalars: $\phi_*: \text{Mod}(B) \to \text{Mod}(B)$, sends some $N$ to the abelian group $N$ by $a \cdot n = \phi(a) \cdot n$\\
	\indent (3) Coextension of Scalars: $\phi = \text{Hom}_A(B, -): \text{Mod}(A) \to \text{Mod}(B)$.
	If $M$ is an $A$-module, then the $B$-mod str on $\text{Hom}_A(B,m)$ is given by 
	\[(b \cdot f)(x) = f(bx)\]
We can check that there are adjunctions, extension of scalars is left adjoint to restriction of scalars which is left adjoint to coextension of scalars.
\end{example}

\subsection{Theorems about adjoints}

\begin{theo}{}{}
	Left/right adjoints are unique up to unique iso.
\end{theo}
Comment: Constructions of sheafification.

\begin{theo}{(Adjoints compose)}{}
	Given adjunctions $C \rightleftarrows D \rightleftarrows E$, then the composite of the left adjoints is left adjoint to the composite of the right adjoints.
\end{theo}
\begin{proof}
	We have 
	\[\text{Hom}_E(L'Lc, e) \cong \text{Hom}_D(Lc, R'e) \cong \text{Hom}_C(c, RR'e)\]
	and natural isomorphisms compose.
\end{proof}

\begin{theo}{(RAPL) dually (LAPC)}{}
Right adjoints preserve limits and dually left adjoints preserve colimits.	
\end{theo}

\begin{coroll}{}{}
	$M \otimes_A (-): \text{Mod}(A) \to \text{Mod}(A)$ preserves colimits, in particular
	\[M \otimes_A \bigoplus_i N_i \cong \bigoplus_i (M \otimes_A N_i)\]
\end{coroll}

\begin{definition}
	$M \in \text{Mod}(A)$ is flat if the following equivalent conditions hold:\\
	\indent (1) $M \otimes_A (-): \text{Mod}(A) \to \text{Mod}(A)$ preserves finite limits\\
	\indent (2) This functor is "left exact:", preserves injections after tensoring.
\end{definition}

\end{document}
