\documentclass{article}

\usepackage[english]{babel}
\usepackage{amsmath}
\usepackage{amssymb}
\usepackage{amsthm}
\usepackage[letterpaper,top=2cm,bottom=2cm,left=3cm,right=3cm,marginparwidth=1.75cm]{geometry}
\usepackage{graphicx}
\usepackage[colorlinks=true, allcolors=blue]{hyperref}
\usepackage{fancyhdr}
\usepackage{tikz}
\usetikzlibrary{matrix}
\usepackage[most]{tcolorbox}
\usepackage{hyperref}
\usepackage{array}

\theoremstyle{definition}
\newtheorem{example}{Example}[section]

\theoremstyle{definition}
\newtheorem{definition}{Definition}[section]

\theoremstyle{remark}
\newtheorem*{remark}{Remark}

\newtcbtheorem{theo}%
	{Theorem}{colback = blue!5!white,colframe = blue!50!black!50!}{theorem}
	
\newtcbtheorem{lemm}%
	{Lemma}{colback = green!5!white,colframe = green!50!black!50!}{lemma}

\newtcbtheorem{clai}%
{Claim}{colback = blue!5!white,colframe = blue!50!black!50!}{claim}

\newtcbtheorem{coroll}%
	{Corollary}{colback = purple!5!white,colframe = purple!50!black!50!}{corollary}    

\newtheorem{theorem}{Theorem}[section]
\newtheorem{corollary}{Corollary}[theorem]
\newtheorem{lemma}[theorem]{Lemma}
\newtheorem{claim}[theorem]{Claim}
\newcommand{\R}{\mathbb{R}}
\newcommand{\C}{\mathbb{C}}
\newcommand{\Z}{\mathbb{Z}}
\newcommand{\N}{\mathbb{N}}
\newcommand{\Q}{\mathbb{Q}}
\newcommand{\mb}[1]{\mathbb{#1}}
\newcommand{\mc}[1]{\mathcal{#1}}
\newcommand{\mk}[1]{\mathfrak{#1}}
\newcommand{\un}{\cup}
\newcommand{\ic}{\cap}
\pagestyle{fancy}
\newcommand\size{1}% distance of nodes from center

\begin{document}
\textbf{Today: More on the topology of the Spec}
Recall last time we defined Spec($A$) as well as the closed sets of the topology on it and 
\[V(S) = V(\langle S \rangle) \quad ,\quad V(I) = V(\text{rad}(I))\]
\begin{definition}
	A ring $A$ is a discrete valuation ring (DVR) if $A$ is a PID and has a unique nonzero prime ideal $\mk{m}$.
\end{definition}

Name comes from number theory, 10 or so equivalent definitions on it. 
Kind of the simplest thing next to a field, with only prime ideal, Spec $(A)$ is rather simple. 
Note that it is also a local ring

\begin{example}
	$(A, \mk{m})$ DVR, primes are $\langle 0 \rangle>$ and $\mk{m}$. Spec($A$) has two elements, with only one closed point, the Sierpinski Space. 
	Note that it is not Hausdorff.
	Exercise in PSet 4, Spec($A$) is Hausdorff $\Longleftrightarrow$ every prime is maximal
\end{example}

\begin{example}
	$K$ and algebraically closed field, primes of $k[t]$ are all of the form $\langle 0 \rangle$ or $(t - \alpha)$ for $\alpha \in K$.
\end{example}

\begin{example}
	$S$ a set, $\{K_s\}_{s \in S}$ a set of fields indexed by $s$.
	Then there's a homeomorphism
	\[\text{Spec}(\prod_{s \in S}K_s) \cong \beta(S)\]
	Where $\beta$ is the Cech-Stone compactification of $S$ with the discrete topology, in bijection with the set of ultrafilters on $S$.	
\end{example}

\textbf{Recall:} Given $f \in A$, $D(f) = \text{Spec}(A)/V(f)$ or the equivalent quotient definition.
Note that from the definition of a prime ideal
\[fg \notin \mk{p} \Longleftrightarrow [f \notin \mk{p} \text{ and } g \notin \mk{p}]\]
So
\[ D(f) \ic D(g) = D(fg)\]

\begin{lemm}{}{}
	Given elements $\{g_{\lambda}\}_{\lambda \in \delta}$ and $f$ of $A$,
	\[D(f) \subset \bigcup_{\lambda \in \delta} D(g_{\lambda}) \text{ iff } f \in \text{rad}(I)
	\]
\end{lemm}

\begin{proof}
	\[D(f) \subset \bigcup_{\lambda \in \delta} D(g_{\lambda}) \Longleftrightarrow V(f) \supset V(I) = V(\text{rad}(I))	\]
whcih is true iff 
\[\text{rad(f)} \subset \text{rad(I)} \Longleftrightarrow \langle f \rangle \subset \text{rad}(I)\] 
\end{proof}

Apparently big relation between logic and commutative algebra/ alg geo, a recent advancement echoing this is the Andre-Ort Conjecture

\begin{coroll}{}{}
	\[\text{Spec}(A) = \bigcup_{\lambda \in \delta} D(g_{\lambda}) \text{ iff } \langle g_{\lambda} \, |\, \lambda \in \delta\rangle = A\]
\end{coroll}

\textbf{AG terminology: } A topological space $X$ is quasicompact if every open cover of $X$ has a finite subcover.
Especially in France and Bourbaki, compact would normally mean quasicompact and Hausdorff.\\

$A$ a ring. Then \\
\indent (1) The opens $\{D(f)\}_{f \in A}$ form a basis for the topology on Spec($A$).\\
\indent (2) Each $D(f)$ is quasicompact.\\
\indent (3) $\text{Spec}(A) = D(1)$ is compact.\\

Proof of (1): We have 
	\[V(I) = \bigcap_{f \in I} V(f) \Longrightarrow D(I) = \bigcup_{f \in I} D(f)\]

	Proof of (2): 

\section{Quasiseparatedness}

\begin{definition}
	A top space $X$ is quasiseparated if forall qc opens $U,V \subset X$, then $U \ic V$ is also quasicompact.\\

	Lemma: $X$ a top space. Assume there exists a basis $\beta$ of $X$ st.\\
	(1) Every $b \in \beta$ is quasicompact.\\
	(2) $\beta$\\
Then $X$ is quasiseparated.
\end{definition}

\begin{coroll}{}{}
 For any ring $A$, Spec($A$) is qcqs.
\end{coroll}

\section{Specializations}

\begin{definition}
	$X$ a topological space.\\
	(1) Given $s, \eta \in S$ we say that $s$ is a specialization of $\eta$ and $\eta$ is a generization of $s$ if $s \in \overline{\{\eta\}}$. 
	Often written s squigglyline to $s$ $\eta$\\
	(2) $\eta \in X$ is generic if $\overline\{\eta\} = X$.
\end{definition}

Have a predorder w/ underlying set $X$ and $s \leq \eta$ of $s \in \overline{\{\eta\}}$
Since $\overline{\mk{p}} = \{\mk{q} \,|\, \mk{q} \supset \mk{p}\}$ so this $\leq$ is a partial order on $\text{Spec}(A)$\\

\section{Irreducibility}

\begin{definition}
	$X$ a nonempty topological space. $X$ is irreducible if $X$ cannot be written as a 
	\[X = Z_1 \un Z_2 \]
\end{definition}
with $Z_1,Z_2$ proper closed sets.

\begin{lemm}{}{}
	$X \neq \emptyset$, TFAE:\\
	(1) $X$ is irreducible\\
	(2) $\forall U,V \subset X$, nonempty opens, $U \ic V \neq \emptyset$.\\
	(3) Every nonempty $U \subset X$ is dense.\\
	(4) Every nonempty open of $X$ is connected.\\
	(5) Every nonempty open of $X$ is irreducible.
\end{lemm}

\begin{lemm}{}{}
	$S \subset X$ is irreducible iff $\overline{S} \subset X$ is irreducible.
\end{lemm}

\begin{lemm}{}{}
	$(S_{\lambda})_{\lambda \in \delta}$ an increasing chain of irred subsets of $X$.
	Then $\bigcup_{\lambda \in \delta} S_{\lambda}$ is irreducible.
\end{lemm}

Note tha above and Zorn's says that every irred subset is contained in a maximal one.

\begin{lemm}{}{}
	The map Spec(A) to irred closed subsets, sending $\mk{p} \to \overline{\{\mk{p}\}} = V(\mk{p})$ is a bijection. 
	As a consequence,\\
	(1) Every irred closed subset of Spec($A$) has a unique generic point (this is called being sober)\\
	(2) minimal primes under this correspond to irreducible components.\\

	Proof the first is immediate from the bijection.
\end{lemm}



\end{document}
