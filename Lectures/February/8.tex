\documentclass{article}

\usepackage[english]{babel}
\usepackage{amsmath}
\usepackage{amssymb}
\usepackage{amsthm}
\usepackage[letterpaper,top=2cm,bottom=2cm,left=3cm,right=3cm,marginparwidth=1.75cm]{geometry}
\usepackage{graphicx}
\usepackage[colorlinks=true, allcolors=blue]{hyperref}
\usepackage{fancyhdr}
\usepackage{tikz}
\usetikzlibrary{matrix}
\usepackage[most]{tcolorbox}
\usepackage{hyperref}
\usepackage{array}

\theoremstyle{definition}
\newtheorem{example}{Example}[section]

\theoremstyle{definition}
\newtheorem{definition}{Definition}[section]

\theoremstyle{remark}
\newtheorem*{remark}{Remark}

\newtcbtheorem{theo}%
	{Theorem}{colback = blue!5!white,colframe = blue!50!black!50!}{theorem}
	
\newtcbtheorem{lemm}%
	{Lemma}{colback = green!5!white,colframe = green!50!black!50!}{lemma}

\newtcbtheorem{clai}%
{Claim}{colback = blue!5!white,colframe = blue!50!black!50!}{claim}

\newtcbtheorem{coroll}%
	{Corollary}{colback = purple!5!white,colframe = purple!50!black!50!}{corollary}    

\newtheorem{theorem}{Theorem}[section]
\newtheorem{corollary}{Corollary}[theorem]
\newtheorem{lemma}[theorem]{Lemma}
\newtheorem{claim}[theorem]{Claim}

\pagestyle{fancy}
\newcommand\size{1}% distance of nodes from center
\newcommand{\R}{\mathbb{R}}
\newcommand{\C}{\mathbb{C}}
\newcommand{\Z}{\mathbb{Z}}
\newcommand{\N}{\mathbb{N}}
\newcommand{\Q}{\mathbb{Q}}
\newcommand{\mb}[1]{\mathbb{#1}}
\newcommand{\mc}[1]{\mathcal{#1}}
\newcommand{\mk}[1]{\mathfrak{#1}}
\newcommand{\un}{\cup}
\newcommand{\ic}{\cap}
\begin{document}

\textbf{Last Time:} Finished up everything we will talk about Spec for the time being, next homework will still be about the properties of Spec.
Next few lectures will focus on Category theory/review, plugged his friend Riehl's book (goated book btw)\\

Question from Vihaan, is there a characterization of the Spec map from an inclusion of rings?
Example, take $A$ an integral domain and consider its inclusion into $\text{Frac}(A)$, then $\text{Spec}(B) \to \text{Spec}(A)$ just picks out the 'generic' point.
Also in general, inclusion of number rings can be very complicated.

\section{Category Theory}
We will first not worry about Set theory paradoxes, this won't effect our intuition and is easily solved (i.e. considering locally small categories or not worrying about the category of sets)

\begin{definition}
	A category $C$ consists of:\\
	\indent (0) A collection Obj(C) of objects of $C:X,Y,Z$\\
	\indent (1) A collection of morphisms $f,g$ so that each morphism has a specified source(domain) and the target (codomain) $f: X \to Y$\\
	\indent For each object $X$ there's a specified identity $\text{id}_x: X \to X$\\
\indent Given $f: X \to Y$ and $g: Y \to Z$ ther;s a specified composite $gf: X \to Z$ that is\\
\indent Unital: $f \text{id}_x = f = \text{id}_yf$\\
\indent Associative $(hg)f = h(gf)$ for any composable homomorpshims.
\end{definition}

We will often denote $\text{Hom}(X,Y)$ or $C(X,Y)$ the set of morphisms $f: X \to Y$.

\begin{example}
	We have the "concrete" categories, i.e. Set, Top, Grp, Ring, Mod(A) with the "obvious" morphisms(all functions, continuous fns, group homs, ring homs, A-mod homs) respectively.
\end{example}

\begin{example}
	$(P, \leq)$ a poset category with $\text{Hom}_p(p,q)$ equal to $*$ (a unique morphism) when $p \leq q$ and none otherwise.
\end{example}

\begin{definition}
	A map $f:X \to Y$ is an isomoprhims if there exists a $g: Y \to X$ st
	\[gf = \text{id}_x \text{ and } fg = \text{id}_y\]
\end{definition}

\begin{example}
In Set, Grp, Ring, Mod(A) an isomorphism is precisely a morphism that is bijective.	
\end{example}

However, there is the standard nonexample in Top, where 
\[[0,1) \to S^1 \quad : \quad x \to e^{2 \pi i x}\]
which is a continuous bijection with no inverse.

How do we define a relation (or even a morphism) between two categories

\begin{definition}
	For $C,D$ categories, a functor $F: C \to D$ is \\
\indent (0) an assignment $F: \text{Obj}(c) = \text{Obj}(d)$\\
\indent An assignment on maps $[f:X \to Y] \to [F(f): F(X) \to F(Y)]$ satisfying\\
\[F(gf) = F(g)F(g) \quad, \quad F(\text{id}_x) = \text{id}_{F(x)}\]

\end{definition}

F
\begin{definition}
	For $C$ a category, the opposite category of $C$ is the category $C^{\text{op}}$ with\\
	\indent (0)$\text{Obj}(C^{\text{op}}) = \text{Obj}(C)$\\
	\indent All the arrows are "reversed", along with the sens in composing.
\end{definition}

\begin{example}
	We have a contravariant functor $\text{Spec}: \text{Ring}^{\text{op}} \to \text{Top}$
	as was defined earlier when discussing the Spectrum, where we showed functoriality
\end{example}

\begin{example}
	We have a class of Forgetful Functors (Group, Ring, A-mod) where we simply forget the extra structure.
\end{example}

\begin{example}
	$C$ any category, there is a functor $\text{Hom}_C(-,-): C^{op} \times C \to \text{Set}$
	which takes $(X,Y) \to \text{Hom}_C(X,Y)$
\end{example}

\begin{example}
	$A$ a ring, the above functor refines to $\text{Mod}(A)^{op} \times \text{Mod}(A) \to \text{Mod}(A)$
\end{example}

One may ask if there are functors that add structure, well the answer is yes as you probably guessed.
Consider the functors $\text{Set} \to \text{Top}$ which sends a set $X$ to the topological space with the discrete and indiscrete topolgy, these are respectively left and right adjoint to the forgetful functor on  Top.\\

Brief tangent on how left/right adjoints preserve colimits/limits, but this will be elaborated on in future lectures.

\begin{definition}
	$F,G: C \to D$ A natural transformation $\alpha: F \Rightarrow G$ or $\alpha: F \to G$ consists of morphsism $\alpha(x): F(x) \to F(x)$ for all $x \in \text{Obj}(C)$ such that the following commutative diagram commutes.
\end{definition}
We observe that natural transformations compose by "vertical stacking".

\begin{example}
	$K$ a field, the standard example (at least fomr Riehl) is that there is a natural transformation from $k$-Vector-spaces to itself by taking the "double dual".
	This also applies construction also applies to more general $A$-modules. 
	However, this isomorphism from finite dimensional in Vector spaces needs to be replaced by finite projective modules.
\end{example}

 \begin{definition}
 	A natural transformation $\alpha: F \to G$ is called a natural isomorphism if each component is an isomorphism, or categorically if there it is an isomorphism in the category of natural transformations
 \end{definition}
 
\end{document}
