\documentclass{article}

\usepackage[english]{babel}
\usepackage{amsmath}
\usepackage{amssymb}
\usepackage{amsthm}
\usepackage[letterpaper,top=2cm,bottom=2cm,left=3cm,right=3cm,marginparwidth=1.75cm]{geometry}
\usepackage{graphicx}
\usepackage[colorlinks=true, allcolors=blue]{hyperref}
\usepackage{fancyhdr}
\usepackage{tikz}
\usetikzlibrary{matrix}
\usepackage[most]{tcolorbox}
\usepackage{hyperref}
\usepackage{array}

\theoremstyle{definition}
\newtheorem{example}{Example}[section]

\theoremstyle{definition}
\newtheorem{definition}{Definition}[section]

\theoremstyle{remark}
\newtheorem*{remark}{Remark}

\newtcbtheorem{theo}%
	{Theorem}{colback = blue!5!white,colframe = blue!50!black!50!}{theorem}
	
\newtcbtheorem{lemm}%
	{Lemma}{colback = green!5!white,colframe = green!50!black!50!}{lemma}

\newtcbtheorem{clai}%
{Claim}{colback = blue!5!white,colframe = blue!50!black!50!}{claim}

\newtcbtheorem{coroll}%
	{Corollary}{colback = purple!5!white,colframe = purple!50!black!50!}{corollary}    

\newtheorem{theorem}{Theorem}[section]
\newtheorem{corollary}{Corollary}[theorem]
\newtheorem{lemma}[theorem]{Lemma}
\newtheorem{claim}[theorem]{Claim}
\newcommand{\R}{\mathbb{R}}
\newcommand{\C}{\mathbb{C}}
\newcommand{\Z}{\mathbb{Z}}
\newcommand{\N}{\mathbb{N}}
\newcommand{\Q}{\mathbb{Q}}
\newcommand{\mb}[1]{\mathbb{#1}}
\newcommand{\mc}[1]{\mathcal{#1}}
\newcommand{\mk}[1]{\mathfrak{#1}}
\newcommand{\un}{\cup}
\newcommand{\ic}{\cap}
\pagestyle{fancy}
\newcommand\size{1}% distance of nodes from center

\begin{document}
\textbf{Today: More on localization, the spectrum of a ring}.\\

\textbf{Recall:} $\mk{p} \subset A$ prime, we write $A_{\mk{p}} := A/[(A/ \mk{p})^{-1}]$.
We observe that $A_{\mk{p}}$ has a unique maximal ideal: $\mk{p} A_{\mk{p}}$, we prove this by noting the bijection between primes not intersecting $A/\mk{p}$ and primes in the localization $A_{\mk{p}}$.\\

\textbf{Notation:} The residue field of $\mk{p}$ is $\kappa(\mk{p}) = A_{\mk{p}}/\mk{p}A_{\mk{p}}$

\begin{definition}
	A ring $A$ is local if $A$ has a unique maximal ideal $\mk{m}$, often  "let $(A, \mk{m})$ be a local ring". ETA The Zariski Topology will explain why these are local. \\

	Here's a small digression about total quotient rings.
\begin{definition}
	The totla quotient ring of a ring $A$ is 
	\[\text{Quot}(A) := A[\text{nzd}(A)^{-1}]\]
\end{definition}

\textbf{Observation:} $A$ is a domain iff $\text{nzd}(A) = A/\{0\}$, so $\text{Quot}(A)$ is a field as we inverted all nonzero elements.
This is often called the fraction field of $A$ and denoted by $\text{Frac}(A)$.
Moreover, in this case, the universal property mpa $A \to \text{Quot}(A)$ is injective.\\

\textbf{UP of Fraction Fields} $A$ a domain, For any injection $\phi: A \hookrightarrow K$ where $K$ is afield there is a unique extension
\[\overline{\phi}: \text{Frac}(A) \to K\]
that commutes with the map $A \to \text{Quot}(A)$
In other words, "the fraction field is the smallest field that contains $A$".\\

\textbf{Note:} For $\mk{p} \subset A$ prime, or $A/\mk{p}$ is a domain, So $\text{Quot}(A/\mk{p})$ is a field. 
This begs a question, how do the two fields associated with $\mk{p}$, $A_{\mk{p}}/\mk{p}$ and Quot()
$A/\mk{p}$ compare?\\

\textbf{Proposition: }$\mk{p} \subset A$ prime.
There is a unique iso Quot($A/\mk{p}) \to A_{\mk{p}}/\mk{p}$ fitting into a commutative diagram.\\

\begin{proof}
	\text{Quot}($A/\mk{p}$) is the universal $A$-algebra in which $\mk{p}$ is killed and $A/\mk{p}$ becomes invertible.\\

	On the other hand, $A_{\mk{p}}/\mk{p}$ is the universal $A-$algebra in which $A/\mk{p}$ becomes invertible and $\mk{p}$ is killed. 
	These two objects have the same universal property and so are isomorphic.
\end{proof}

 
\section{Spectrum of a Ring}

The geometry of rings, we want to study "spaces cut out by solutions to systems of polynomial equations".
\begin{example}
We have the cuspidal cubic	$V(y^2 - x^3) := \{(x,y) \in \C^2 \,|\, y^2 - x^3 = 0\}$, the nodal cubic is $V(y^2 - (x^3 - x^3))$
\end{example}

The idea is that given some polynomial $f \in \C[x_1, \dots, x_n]$ we should study the ring
\[\C[x_1, \dots, x_n]/\langle f \rangle\]
where $f = 0$ is universally solved.
In whatever "space" is associated to $\C[x_1, \dots, x_n]$, we should have closed vanishing sets associated to any polynomial $f \in \C[x_1, \dots, x_n]$.\\

From an algebraic geo point of view, it might be easier to define the topology via the closed sets as opposed to the open sets which is usually standrard.
I won't give the definition here since it's rather standard, just take the complements with the standard open set definition.\\

\begin{definition}
	$A$ a ring \\
	\indent $\text{Spec}(A) := \{\text{primes } \mk{p} \subset A\}$\\
	\indent $S \subset A$, the vanishing set of $S$ is \[V(S) = \{\mk{p} \in \text{Spec}(A) \, |\, \mk{p} \supset S\} = \{\mk{p} \, |\, \text{S is annhilated by  $A \to A/\mk{p}$
}\}\]
\end{definition}

Observation: $V(S) = V(\langle S \rangle)$.\\
We shall denote Idl($A$) as the set of ideals of $A$.\\
\textbf{Proposition:} The map 
\[(\text{Idl}(A), \subset) \to (\text{sub(Spec}(A))), \subset\]
reverses inclusion. \\

Now note that \\
\indent (1) $V(0) = \text{Spec}(A), \quad V(1) = \emptyset$\\
\indent (2) Given ideals $\{I_s\}_{s\in S}$, then 
\[V(\bigcup I_s) = V(\sum I_s) = \bigcap_{s \in S} I_s\]
\indent $I,J \subset A$ ideals, then 
\[V(I \ic J) = V(IJ) = V(I) \un V(J)\]

\begin{definition}
	The Zariski Topology on Spec($A$) is the topology with the closed sets equal to $\{V(I), I \in \text{Idl}(A)\}$ which defines a topology as demonstrated above.
\end{definition}

\begin{lemm}{}{}
	For $\mk{p} \subset A$ a prime, and $I_1, \dots, I_n \subset A$ ideals, TFAE\\
	\indent (1) $\mk{p} \supset I_1\dots I_n$\\
	\indent (2) $p \supset I_1 \ic \dots \ic I_n$\\
	\indent (3) $\mk{p}$ contains some $I_k$.
\end{lemm}

Question: Can we go back from a closed subset of Spec$(A)$ and go back to an ideal of $A$.
Yes, this leads us the definition of a radical.
By definition, 
\[V(I) = V(\bigcap_{\mk{p} \supset I}\mk{p}) = V(r(I))\]
Big word: Scheinnullstellenstatz, for an ideal $I \subset A$
\[\bigcap_{\mk{p} \supset I}\mk{p} = \{a \in A \, | \, \exists n > 0, a^n \in I\}\]
Other notations for the radical are either $\text{rad}(I), \sqrt{I},$ or $\mk{r}(I)$, and $I$ is radical if $\mk{r}(I)  = I$.\\

Note that prime implies radical, and that the radical operation is idempotent.
Also $I \subset A$ is radical iff $I$ is some intersection of primes.

\begin{example}
	$\langle n \rangle \subset \Z$ is radical iff $n$ is squarefree.
\end{example}

We then have that \\
\indent (1) $rad(I(s)) = I(s)$, i.e. $I(s)$ is radical.\\
\indent (2) $I(V(a)) = rad(a) \rightarrow V(a) = V(rad(a))$\\
\indent (3) $V(I(s)) = \overline{S}$\\
\indent (4) there is abijection between radical ideals in Idl$(A)$ and closed sets in Spectrum(A).






























	
	
	
	
\end{definition}










\end{document}
