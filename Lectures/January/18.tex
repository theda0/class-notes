\documentclass{article}

\usepackage[english]{babel}
\usepackage{amsmath}
\usepackage{amssymb}
\usepackage[letterpaper,top=2cm,bottom=2cm,left=3cm,right=3cm,marginparwidth=1.75cm]{geometry}
\usepackage{graphicx}
\usepackage[colorlinks=true, allcolors=blue]{hyperref}
\usepackage{fancyhdr}
\usepackage{tikz}
\usetikzlibrary{matrix}
\usepackage[most]{tcolorbox}
\usepackage{hyperref}
\usepackage{array}

\newtcbtheorem{theo}%
	{Theorem}{colback = blue!5!white,colframe = blue!50!black!50!}{theorem}
	
\newtcbtheorem{lemm}%
	{Lemma}{colback = green!5!white,colframe = green!50!black!50!}{lemma}

\newtcbtheorem{clai}%
{Claim}{colback = blue!5!white,colframe = blue!50!black!50!}{claim}

\newtcbtheorem{coroll}%
	{Corollary}{colback = purple!5!white,colframe = purple!50!black!50!}{corollary}    


\newtheorem{theorem}{Theorem}[section]
\newtheorem{corollary}{Corollary}[theorem]
\newtheorem{lemma}[theorem]{Lemma}
\newtheorem{claim}[theorem]{Claim}

\pagestyle{fancy}
\newcommand\size{1}% distance of nodes from center
\newcommand{\R}{\mathbb{R}}
\newcommand{\C}{\mathbb{C}}
\newcommand{\Z}{\mathbb{Z}}
\newcommand{\N}{\mathbb{N}}
\newcommand{\Q}{\mathbb{Q}}
\newcommand{\mb}[1]{\mathbb{#1}}
\newcommand{\mc}[1]{\mathcal{#1}}
\newcommand{\un}{\cup}
\newcommand{\ic}{\cap}
\begin{document}
\textbf{Today: } Quotients, moreon ideals (primes, maximal ideals), most is expected to be review.
HW available after class, due before next Thursday's class.\\

\textbf{Observation:} For any ideal $I \subset A$, you can form the quotient group (under +) $A/I$.
which has a unique ring structure (multiplication) which is exactly what you think it is, which multiplies by representatives of the equivalence classs, i.e.
\[(a+I)(b+I) = ab+I\]
Hence there is a surjective quotient ring homomorphism $q: A \to A/I$.
\textit{Exercise: Check This!}\\

There is a universal property of the quotient, which you can view as a more conceptual formalization of it, as opposed to its explicit construction.
Let $\phi: A \to B$ be a ring hom such that $I \subset I \subset \text{Ker}(\phi)$. Then there is a unique ring homomorphi $\varphi: A/I \to B$ s.t. the following triangle commutes. (The only possible map maps $a+I$ to $\phi(A)$.\\

Alternatively, this also says that there is a bijection between the set of ring homomorphism whose from $A \to B$ whose Kernel contains $I$ with maps from $A/I \to B$.\\

\begin{clai}{}{}
There is a bijection between ideals of $A/I$ and ideals of $A$ containing $I$.
\end{clai}

\textit{Proof: } There are two obvious maps, it is easy to prove they are inverse bijections, this is basically an extension of the corresponding theorem for groups.\\

\begin{lemm}{}{}
Let $A$ be a nonzero ring. Then the following areequivalent.\\
\indent (i) $A$ is a field.\\
\indent (ii) The only ideals of $A$ are $(0)$ and $A$.\\
\indent (iii) All ring homomorphisms from $A$ to a nonzero ring $B$ are injective.
\end{lemm}
\textit{Proof:} $(i) \to (ii):$ It is easy to see that any nonzero ideal must contain $1$, and so is $A$\\

$(ii) \to (iii):$ The kernel of any such ring homomorphism is an ideal $A$. Since $B$ is nonzero, the kernel then must be $(0)$ and so the map must be injective.\\

$(iii) \to (i):$ Since we can always take quotients by ideals, this forces any proper ideal to be $(0)$. 
Hence for any $a \in A/\{0\}$ we see that $(a) = A$ and so we can find an inverse.\\

\textbf{Observation:} If $I$ and $J$ are ideals, $I \un J$ need not be an ideal. Easy to see by taking some examples in $\Z$ (i.e. $(2) \un (3))$
. 
However, we have that the sum of ideals $\{I_s\}_{s \in S}$ is defined as 
\[ \sum_{s \in S} I_s = \langle \bigcup_{s \in S} I_s \rangle\]

Another observation is that the interseciton of any family of ideals is also an ideal, which is trivially verified.\\

Given ideals $I_1, \dots, I_n \subset A$ their product is defined as 
\[I_1 \dots I_n = \langle \{a_1\dots,a_n | a_i \in I_i\}\rangle\]

\textbf{Observation: } $I_1\dots I_n \subset I_1 \cap \dots \cap I_n$. In fact, equality is acheived when these ideasl are pairwise comaximal (by Chinese Remainder Theorem) \\

\textbf{Observation: } Sums and products of ideals are commutative, associative and distributive over each other.\\

Two ideals $A$ and $B$ are comaximal if $A+B = (1)$.
\begin{theo}{Chinese Remainder Theorem (Ring Edition)}{}
Let $I_1, I_2, \dots, I_n$ be ideals of $A$ which are pairwise comaximal.\\ 
\indent (1): $I_1\dots I_n = I_1 \cap \dots \cap I_n$.\\
\indent (2): The natural ring homomorphism
\[A \to A/I_1 \times \dots \times A/I_n\]
is surjective with the kernel of the equal to $I_1\dots I_n = I_1 \cap \dots \cap I_n$\\
\indent (3): The indcued ring homomorphism $A/(I_1 \ic \dots \ic I_n) \to  A/I_1 \times \dots \times A/I_n$ is an isomorphism.
\end{theo}

\textbf{Prime Ideal Definition: } An ideal $\mathfrak{p} \subset A$ if $\mathfrak{p} \neq A$ and if $ab \in \mathfrak{p} \Longrightarrow $ $a \in \mathfrak{p}$ or $b \in \mathfrak{p}$.\\

\textbf{Maximal Ideal Definition: } An ideal $m \in A$ is maximal if $m \neq A$ and all ideals properly continaing $m$ is equal to $A$.\\

\textbf{Examples:\\}
A ring $A$ is a domain if $(0)$ is a prime ideal.\\
In $\Z$, the maximal ideals are all of the form $(p)$ where $p$ is prime, and the prime ideals are $(0)$ and $(p)$.

\begin{clai}{}{}
\indent (i): $I$ is prime iff $A/I$ is a domain.\\
\indent (ii): $I$ is maximal iff $A/I$ is a field.
\end{clai}

\begin{coroll}{}{}
	Maximal ideals are also prime.
\end{coroll}

The bijection between ideals of $A$ containing $I$ and ideals of $A/I$ stays the same when adding the adjectives "prime" and "maximal". Using the third isomorphism theorem where $(A/I)/(J/I)  = A/J$.\\

\begin{lemm}{}{}
Every ring $A$ has a maximal ideal, or equivalently, every ideal $a$ is contained in some maximal (and hence prime) ideal.
\end{lemm}

\textit{Proof:}  This is a standared application of Zorn's Lemma.















 





















\end{document}
